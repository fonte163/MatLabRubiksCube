\documentclass[a4paper,12pt]{scrreprt}

\usepackage[ngerman]{babel}
\usepackage[utf8]{inputenc}
\usepackage{amsthm}
\usepackage{enumerate}
\usepackage{mathtools}
\usepackage{amssymb}
\usepackage{stmaryrd}
\usepackage{listings}
\usepackage[dvipsnames]{xcolor}
\usepackage{hyperref}

\usepackage[headsepline]{scrlayer-scrpage}
\pagestyle{scrheadings}
\clearscrheadfoot
\ohead{Elisa Junghans\\Mirko Dransfeld}
\ihead{Wissenschaftliches Rechnen\\SS 19}
\chead{Praktikum MatLab\\Rubik's Cube}
\cfoot*{\pagemark}

\title{Rubik's Cube}
\subtitle{MatLab Praktikum}
\author{Elisa Junghans\and Mirko Dransfeld}
\date{}

\providecommand*{\elq}{‘}
\providecommand*{\erq}{’}
\providecommand*{\elqq}{“}
\providecommand*{\erqq}{”}

\hypersetup{
    unicode=true,
    colorlinks=true,
    linkcolor=black,
    linktoc=all,
    citecolor=black,
    filecolor=black,
    urlcolor=black,
    final=true
}

\lstset{ %
    basicstyle=\small\ttfamily,
    language=Matlab,
    showtabs=false,
    tabsize=2,
    captionpos=t,
    breaklines=true,
    extendedchars=true,
    showstringspaces=false,
    flexiblecolumns=true,
    numbers=left,
    numbersep=.5em,
    stepnumber=1,
    numberstyle=\color{Gray},
    keywordstyle=\color{Blue},
    commentstyle=\color{ForestGreen}
}

\newcommand{\codeimport}[4][]{
  \lstinputlisting[firstnumber=#2, linerange={#2-#3}#1]{#4.m}
}

\newcommand{\coderef}[1]{
  \hyperref[#1]{\texttt{\nameref{#1}}}
}

\newcommand{\codeinline}[1]{
  \lstinline!#1!
}


\newcommand{\chap}[2]{
  \chapter{#1}\label{#2}
}
\renewcommand{\sec}[2]{
  \section{#1}\label{#2}
}
\newcommand{\subsec}[2]{
  \subsection{#1}\label{#2}
}
\newcommand{\subsubsec}[2]{
  \subsubsection{#1}\label{#2}
}

%%
%%
%%
%%  Hallo Elisa :)
%%
%%
%%
%%  Überschriften sind: (absteigende Wichtigkeit)
%%  \chap{}{}       Kapitel
%%  \sec{}{}        sektion   (ist das ein wort?)
%%  \subsec{}{}     untersektion
%%  \subsubsec{}{}  unteruntersektion
%%
%%  alle funktionieren mit z.B. \chap{<Überschrift>}{<Labelname>}
%%  Die Labelnamen sind dafür um später auf die Funktion verweisen zu können, siehe \coderef
%%  Wichtig: Unterstriche in den Überschrifter escapen mit \_ !
%%
%%
%%
%%  wenn du Code aus einer Datei einbinden willst: \codeimport{<Anfang>}{<Ende>}{<Datei>}
%%  z.B.  \codeimport{44}{49}{RubiksCube}  fügt Zeilen 44-49 aus RubiksCube.m ein
%%
%%  wenn du eine bestimmte Funktion referenzieren willst: \coderef{<Labelname>}
%%  z.B.  \coderef{sec:figure}  macht daraus figure() mit einem Verweis zu \sec{figure()}{figure}. (es setzt den Funktionsnamen ein, so wie du ihn benannt hast)
%%
%%  wenn du einen kleinen inline-codeteil haben willst: \codeinline{<Text>}
%%  z.B.  \codeinline{figure(Option1, Wert1, ...)}  schreibt den <Text> als Code (inline)
%%
%%
%%
%%  Kannst dir ja das angucken was ich bisher geschrieben habe. Bei Problemen oder Feature-Wünschen -> Hey :)
%%
%%
%%

\begin{document}
  \maketitle

  \chap{Aufbau des Programms}{aufbau}
    Zur erleichterten Bearbeitung und Übersichtlichkeit haben wir das Programm in zwei Dateien geteilt. Die Datei \textbf{\nameref{RCube}} beinhaltet das Erstellen der grafischen Oberfläche und die Datei \textbf{\nameref{solveCube}} den Lösungsalgorithmus.

    \sec{RubiksCube.m}{RCube}
      Zu Beginn sei gesagt, dass alles in einer großen Funktion mit sehr vielen \emph{nested functions} geschrieben ist. Dies hat den Grund, dass die MatLab-Version 2015b in einer Datei entweder nur eine Funktion oder ein Skript unterstützt, nicht beides. Dazu sei gesagt, die Version 2018b hat mit solchen Dateien keine Probleme.\\

      In Zeile 3--17 werden globale Variablen gesetzt. Diese werden dann später von anderen Funktionen aufgerufen und verändert. Die Zeilen 20--42 initialisieren Konstanten, wie zum Beispiel die Positionen der Flächen des Würfels im drei-dimensionalen Raum (Zeilen 22--27).

      Dann folgt die Initialisierung des Würfels. Dieser ist in der Variablen \codeinline{face_color_rgb} gespeichert.
      \codeimport{44}{49}{RubiksCube}
      Damit wird für jede der 6 Seiten \codeinline{(i)} und jede der 8 veränderbaren Flächen pro Seite \codeinline{(j)} die zugehörige Farbe aus \codeinline{possible_colors} hinterlegt.

      \subsec{main()}{main}
        Hier wird zuerst die Figur erzeugt, in der anschließend gezeichnet wird. Die Optionen werden in der Form \hyperref[figure]{\codeinline{figure(Option1, Wert1, Option2, Wert2, ...)}} übergeben.
        \codeimport{53}{53}{RubiksCube}
        \begin{itemize}
          \item[\texttt{Name}] `Rubiks Cube' setzt den Namen der Figur
          \item[\texttt{NumberTitle}] `off' verhindert, dass MatLab dem Fenster eine nummerierte Überschrift gibt
          \item[\texttt{MenuBar}] `none' bedeutet, dass keine Menüzeile gezeichnet wird
          \item[\texttt{resize}] `off' verhindert dass Verändern der Größe des Fensters durch den User
        \end{itemize}

        Danach werden mit Hilfe von \coderef{elonUI} und \coderef{azimUI} den Variablen \codeinline{elon} und \codeinline{azim} ihre jeweiligen \emph{Slider} zugewiesen.

        Es folgen Aufrufe von \coderef{uiSetup} und \coderef{generatePatches}, die die restlichen Elemente des UIs anlegen.

      \subsec{ui\_setup()}{uiSetup}
        Hier werden zunächst mit Hilfe von \coderef{patch} die Elemente des UIs erzeugt und, wenn diese später noch verändert werden müssen, globalen Variablen zugewiesen.


      \subsec{get\_elon\_ui()}{elonUI}

      \subsec{get\_azim\_ui()}{azimUI}

      \subsec{generate\_patch\_and\_ui\_menu()}{generatePatches}


\clearpage
    \sec{generate\_solution.m}{solveCube}
      \Large{TODO}


  \chap{besondere MatLab Funktionen}{functions}
    \sec{figure()}{figure}
    \sec{patch()}{patch}
\end{document}
