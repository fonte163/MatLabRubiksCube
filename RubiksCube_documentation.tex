\documentclass[a4paper,12pt]{scrartcl}

\usepackage[ngerman]{babel}
\usepackage[utf8]{inputenc}
\usepackage{amsthm}
\usepackage{enumerate}
\usepackage{mathtools}
\usepackage{amssymb}
\usepackage{stmaryrd}
\usepackage{listings}
\usepackage[dvipsnames]{xcolor}
\usepackage{hyperref}

\usepackage[headsepline]{scrlayer-scrpage}
\pagestyle{scrheadings}
\clearscrheadfoot
\ohead{Elisa Junghans\\Mirko Dransfeld}
\ihead{Wissenschaftliches Rechnen\\SS 19}
\chead{Praktikum MatLab\\Rubik's Cube}
\cfoot*{\pagemark}

\title{Rubik's Cube}
\subtitle{MatLab Praktikum}
\author{Elisa Junghans\and Mirko Dransfeld}
\date{}

\hypersetup{
    unicode=true,
    colorlinks=true,
    linkcolor=black,
    linktoc=all,
    citecolor=black,
    filecolor=black,
    urlcolor=black,
    final=true
}

\lstset{ %
    basicstyle=\small\ttfamily,
    language=Matlab,
    showtabs=false,
    tabsize=2,
    captionpos=t,
    breaklines=true,
    extendedchars=true,
    showstringspaces=false,
    flexiblecolumns=true,
    numbers=left,
    numbersep=.5em,
    stepnumber=1,
    numberstyle=\color{Gray},
    keywordstyle=\color{Blue},
    commentstyle=\color{ForestGreen}
}
\newcommand{\mcode}[3]{
  \lstinputlisting[firstnumber=#1, linerange={#1-#2}]{#3.m}
}

\begin{document}
  \maketitle

  \section{Aufbau des Programms}
    Zur erleichterten Bearbeitung und Übersichtlichkeit haben wir das Programm in zwei Dateien geteilt. Die Datei \textbf{\nameref{sec:aufbau1}} beinhaltet das Erstellen der grafischen Oberfläche und die Datei \textbf{\nameref{sec:aufbau2}} den Lösungsalgorithmus.

    \subsection{RubiksCube.m}\label{sec:aufbau1}
      Zu Beginn sei gesagt, dass alles in einer großen Funktion mit sehr vielen \emph{nested functions} geschrieben ist. Dies hat den Grund, dass die MatLab-Version 2015b in einer Datei entweder nur eine Funktion oder ein Skript unterstützt, nicht beides. Dazu sei gesagt, die Version 2018b hat mit solchen Dateien keine Probleme.
      \newline

      In Zeile 3--17 werden globale Variablen gesetzt. Diese werden dann später von anderen Funktionen aufgerufen und verändert. Die Zeilen 20--42 initialisieren Konstanten, wie zum Beispiel die Positionen der Flächen des Würfels im drei-dimensionalen Raum (Zeilen 22--27).

      Dann folgt die Initialisierung des Würfels. Dieser ist in der Variablen \textbf{face\_color\_rgb} gespeichert.
      \mcode{44}{49}{RubiksCube}
      Damit wird für jede der 6 Seiten (\textbf{i}) und jede der 8 veränderbaren Flächen pro Seite (\textbf{j}) die zugehörige Farbe aus \textbf{possible\_colors} hinterlegt.

      \subsubsection{main()}\label{sec:main}
        Hier wird zuerst die Figur erzeugt, in der anschließend gezeichnet wird. Die Optionen werden in der Form \texttt{figure(Option1, Wert1, Option2, Wert2, \dots)} übergeben.
        \mcode{53}{53}{RubiksCube}
        \begin{itemize}
          \item{\textbf{Name}} `Rubiks Cube' setzt den Namen der Figur
          \item{\textbf{NumberTitle}} `off' verhindert, dass MatLab dem Fenster eine nummerierte Überschrift gibt
          \item{\textbf{MenuBar}} `none' bedeutet, dass keine Menüzeile gezeichnet wird
          \item{\textbf{resize}} `off' verhindert dass Verändern der Größe des Fensters durch den User
        \end{itemize}

        Danach werden mit Hilfe von \textbf{\nameref{sec:elonUI}} und \textbf{\nameref{sec:azimUI}} den Variablen \textbf{elon} und \textbf{azim} ihre jeweiligen \emph{Slider} zugewiesen.

        Es folgen Aufrufe von \textbf{\nameref{sec:uiSetup}} und \textbf{\nameref{sec:generatePatches}}, die die restlichen Elemente des UIs anlegen.

      \subsubsection{ui\_setup()}\label{sec:uiSetup}
        Hier werden zunächst mit Hilfe von \textbf{\nameref{sec:patch}} die Elemente des UIs erzeugt und, wenn diese später noch verändert werden müssen, globalen Variablen zugewiesen.


      \subsubsection{get\_elon\_ui()}\label{sec:elonUI}

      \subsubsection{get\_azim\_ui()}\label{sec:azimUI}

      \subsubsection{generate\_patch\_and\_ui\_menu()}\label{sec:generatePatches}


\newpage
    \subsection{generate\_solution.m}\label{sec:aufbau2}
      \Large{TODO}


  \section{besondere MatLab Funktionen}
    \subsection{patch()}\label{sec:patch}
\end{document}
